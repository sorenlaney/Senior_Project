% Options for packages loaded elsewhere
\PassOptionsToPackage{unicode}{hyperref}
\PassOptionsToPackage{hyphens}{url}
\PassOptionsToPackage{dvipsnames,svgnames,x11names}{xcolor}
%
\documentclass[
  letterpaper,
  DIV=11,
  numbers=noendperiod]{scrartcl}

\usepackage{amsmath,amssymb}
\usepackage{lmodern}
\usepackage{iftex}
\ifPDFTeX
  \usepackage[T1]{fontenc}
  \usepackage[utf8]{inputenc}
  \usepackage{textcomp} % provide euro and other symbols
\else % if luatex or xetex
  \usepackage{unicode-math}
  \defaultfontfeatures{Scale=MatchLowercase}
  \defaultfontfeatures[\rmfamily]{Ligatures=TeX,Scale=1}
\fi
% Use upquote if available, for straight quotes in verbatim environments
\IfFileExists{upquote.sty}{\usepackage{upquote}}{}
\IfFileExists{microtype.sty}{% use microtype if available
  \usepackage[]{microtype}
  \UseMicrotypeSet[protrusion]{basicmath} % disable protrusion for tt fonts
}{}
\makeatletter
\@ifundefined{KOMAClassName}{% if non-KOMA class
  \IfFileExists{parskip.sty}{%
    \usepackage{parskip}
  }{% else
    \setlength{\parindent}{0pt}
    \setlength{\parskip}{6pt plus 2pt minus 1pt}}
}{% if KOMA class
  \KOMAoptions{parskip=half}}
\makeatother
\usepackage{xcolor}
\setlength{\emergencystretch}{3em} % prevent overfull lines
\setcounter{secnumdepth}{-\maxdimen} % remove section numbering
% Make \paragraph and \subparagraph free-standing
\ifx\paragraph\undefined\else
  \let\oldparagraph\paragraph
  \renewcommand{\paragraph}[1]{\oldparagraph{#1}\mbox{}}
\fi
\ifx\subparagraph\undefined\else
  \let\oldsubparagraph\subparagraph
  \renewcommand{\subparagraph}[1]{\oldsubparagraph{#1}\mbox{}}
\fi


\providecommand{\tightlist}{%
  \setlength{\itemsep}{0pt}\setlength{\parskip}{0pt}}\usepackage{longtable,booktabs,array}
\usepackage{calc} % for calculating minipage widths
% Correct order of tables after \paragraph or \subparagraph
\usepackage{etoolbox}
\makeatletter
\patchcmd\longtable{\par}{\if@noskipsec\mbox{}\fi\par}{}{}
\makeatother
% Allow footnotes in longtable head/foot
\IfFileExists{footnotehyper.sty}{\usepackage{footnotehyper}}{\usepackage{footnote}}
\makesavenoteenv{longtable}
\usepackage{graphicx}
\makeatletter
\def\maxwidth{\ifdim\Gin@nat@width>\linewidth\linewidth\else\Gin@nat@width\fi}
\def\maxheight{\ifdim\Gin@nat@height>\textheight\textheight\else\Gin@nat@height\fi}
\makeatother
% Scale images if necessary, so that they will not overflow the page
% margins by default, and it is still possible to overwrite the defaults
% using explicit options in \includegraphics[width, height, ...]{}
\setkeys{Gin}{width=\maxwidth,height=\maxheight,keepaspectratio}
% Set default figure placement to htbp
\makeatletter
\def\fps@figure{htbp}
\makeatother

\KOMAoption{captions}{tableheading}
\makeatletter
\makeatother
\makeatletter
\makeatother
\makeatletter
\@ifpackageloaded{caption}{}{\usepackage{caption}}
\AtBeginDocument{%
\ifdefined\contentsname
  \renewcommand*\contentsname{Table of contents}
\else
  \newcommand\contentsname{Table of contents}
\fi
\ifdefined\listfigurename
  \renewcommand*\listfigurename{List of Figures}
\else
  \newcommand\listfigurename{List of Figures}
\fi
\ifdefined\listtablename
  \renewcommand*\listtablename{List of Tables}
\else
  \newcommand\listtablename{List of Tables}
\fi
\ifdefined\figurename
  \renewcommand*\figurename{Figure}
\else
  \newcommand\figurename{Figure}
\fi
\ifdefined\tablename
  \renewcommand*\tablename{Table}
\else
  \newcommand\tablename{Table}
\fi
}
\@ifpackageloaded{float}{}{\usepackage{float}}
\floatstyle{ruled}
\@ifundefined{c@chapter}{\newfloat{codelisting}{h}{lop}}{\newfloat{codelisting}{h}{lop}[chapter]}
\floatname{codelisting}{Listing}
\newcommand*\listoflistings{\listof{codelisting}{List of Listings}}
\makeatother
\makeatletter
\@ifpackageloaded{caption}{}{\usepackage{caption}}
\@ifpackageloaded{subcaption}{}{\usepackage{subcaption}}
\makeatother
\makeatletter
\@ifpackageloaded{tcolorbox}{}{\usepackage[many]{tcolorbox}}
\makeatother
\makeatletter
\@ifundefined{shadecolor}{\definecolor{shadecolor}{rgb}{.97, .97, .97}}
\makeatother
\makeatletter
\makeatother
\ifLuaTeX
  \usepackage{selnolig}  % disable illegal ligatures
\fi
\IfFileExists{bookmark.sty}{\usepackage{bookmark}}{\usepackage{hyperref}}
\IfFileExists{xurl.sty}{\usepackage{xurl}}{} % add URL line breaks if available
\urlstyle{same} % disable monospaced font for URLs
\hypersetup{
  pdftitle={Windows Functions 101: What you need to know},
  pdfauthor={Soren Laney},
  colorlinks=true,
  linkcolor={blue},
  filecolor={Maroon},
  citecolor={Blue},
  urlcolor={Blue},
  pdfcreator={LaTeX via pandoc}}

\title{Windows Functions 101: What you need to know}
\author{Soren Laney}
\date{2022-11-04}

\begin{document}
\maketitle
\ifdefined\Shaded\renewenvironment{Shaded}{\begin{tcolorbox}[enhanced, frame hidden, borderline west={3pt}{0pt}{shadecolor}, interior hidden, boxrule=0pt, breakable, sharp corners]}{\end{tcolorbox}}\fi

\hypertarget{windows-overview}{%
\subsection{Windows Overview}\label{windows-overview}}

\includegraphics{Windows.png}

Windows functions are great for visualizing data and extracting insights
that cannot be gathered through grouby and multi groupby fuinctions. To
further understand this we will use the following dataset to compare the
differnece and function of groupby and windows functions. This is an
example dataset that shows how many hours that employees clocked during
a week (it was a slow week). The example code will also be given in
pyspark.

\begin{longtable}[]{@{}lll@{}}
\toprule()
employee & branch & Hrs \\
\midrule()
\endhead
Jake Ballard & 1 & 10 \\
Olivia Pope & 1 & 19 \\
Fitz Gerald & 1 & 8 \\
Cyrus Bean & 1 & 9 \\
Melly Grant & 2 & 16 \\
Harrison Right & 2 & 15 \\
Elizabeth North & 2 & 9 \\
Quin Perkins & 3 & 15 \\
Aby Wheelend & 3 & 12 \\
Markus Walker & 3 & 6 \\
\bottomrule()
\end{longtable}

\#\#\#\# Groupby roupby functions opperate by summarizing data according
to a specified group. I this example we will grouby the hours worked in
each brand and then sumarize that data by taking the average of each
group. This is often much faster than a windows due to the simplicity
and is avaliability in most programing languages. A groupby would yield
the following result from running the following code.

\begin{verbatim}
df.groupBy("branch").avg("Hrs").show(truncate=False)
\end{verbatim}

\begin{longtable}[]{@{}ll@{}}
\toprule()
branch & Avg Hrs \\
\midrule()
\endhead
1 & 11.5 \\
2 & 13.3 \\
3 & 11 \\
\bottomrule()
\end{longtable}

\hypertarget{windows}{%
\paragraph{Windows}\label{windows}}

When preforming a windows function you can attain aggragate functions
without loosing the the rows. This can be very useful when trying to
create a new dataframe. An example of how this dataframe could provide
more value than a groupby is if you wanted to display a double bar chart
comparing the average hours worked in a branch to that of the hours
worked by an individual employee. This function has two main parts, a
partition and a relational grouped dataset. The first term is what it
will partition the dataset into to create the new term/column. The
second term can be one of couple of functions that can transform the
data. You can then create a new column aggregating the function with
that ``window'' that was created as demonstrated below. A windows
functions yields the following result

\begin{verbatim}
Window.partitionBy("branch").orderBy("hrs")
df.withColumn("Avg Hrs",avg.over(windowSpec))
df.withColumn("row_number",row_number.over(windowSpec))
\end{verbatim}

\begin{longtable}[]{@{}llll@{}}
\toprule()
employee & branch & Hrs & Avg Hrs \\
\midrule()
\endhead
Quin Perkins & 3 & 15 & 11 \\
Aby Wheelend & 3 & 12 & 11 \\
Markus Walker & 3 & 6 & 11 \\
Olivia Pope & 1 & 19 & 11.5 \\
Jake Ballard & 1 & 10 & 11.5 \\
Fitz Gerald & 1 & 8 & 11.5 \\
Cyrus Bean & 1 & 9 & 11.5 \\
Melly Grant & 2 & 16 & 13.3 \\
Harrison Right & 2 & 15 & 13.3 \\
Elizabeth North & 2 & 9 & 13.3 \\
\bottomrule()
\end{longtable}

\hypertarget{history}{%
\subsection{History}\label{history}}

The windows function originates as a SQL function, however has made it
way into many other programming languages over the last couple of years.
This function was introduced in 2003. Added functionalty was introduced
later on.

\hypertarget{example-code}{%
\subsection{Example Code}\label{example-code}}

\hypertarget{sql}{%
\paragraph{SQL}\label{sql}}

\begin{verbatim}
SELECT  
        RANK() OVER (PARTITION BY Value_1 ORDER BY Value_2 DESC) 
            AS Value_1_ranking,
        Column_1,
        Column_2, 
        Column_3, 
        Column_4
FROM df;
\end{verbatim}

\hypertarget{pyspark}{%
\paragraph{Pyspark}\label{pyspark}}

\begin{verbatim}
Window.partitionBy("value_1").orderBy("value_2")
\end{verbatim}

\begin{center}\rule{0.5\linewidth}{0.5pt}\end{center}

\hypertarget{other-articles-on-windows-function}{%
\subsection{Other Articles on Windows
Function}\label{other-articles-on-windows-function}}

\hypertarget{blog-articles}{%
\paragraph{Blog Articles}\label{blog-articles}}

\begin{itemize}
\tightlist
\item
  \href{\textquotesingle{}https://github.com/byuibigdata/spark_guide/blob/main/aggregate_calculations.md\textquotesingle{}}{BYUI
  Big Data Windows}
\item
  \href{\textquotesingle{}https://drill.apache.org/docs/sql-window-functions-introduction/\#:~:text=Window\%20functions\%20operate\%20on\%20a,to\%20calculate\%20the\%20returned\%20values.\textquotesingle{}}{Apache
  Drill Article}
\item
  \href{\textquotesingle{}https://towardsdatascience.com/a-guide-to-advanced-sql-window-functions-f63f2642cbf9\textquotesingle{}}{SQL
  Windows Functions}
\item
  \href{\textquotesingle{}https://en.wikipedia.org/wiki/Window_function_(SQL)\textquotesingle{}}{Wikipedia
  on SQL}
\item
  \href{\textquotesingle{}https://www.geeksforgeeks.org/window-functions-in-sql/\textquotesingle{}}{Geeks
  for Geeks SQL Windows Function}
\end{itemize}

https://www.geeksforgeeks.org/window-functions-in-sql/

\hypertarget{docuementations}{%
\paragraph{Docuementations}\label{docuementations}}

\begin{itemize}
\tightlist
\item
  \href{\textquotesingle{}https://sparkbyexamples.com/spark/spark-sql-window-functions/\textquotesingle{}}{Pyspark
  Docuementation}
\item
  \href{\textquotesingle{}https://mode.com/sql-tutorial/sql-window-functions/\textquotesingle{}}{SQL
  DOcuementation(Mode)}
\item
  \href{\textquotesingle{}https://cran.r-project.org/web/packages/dplyr/vignettes/window-functions.html\textquotesingle{}}{R
  Docuementation}
\end{itemize}



\end{document}
